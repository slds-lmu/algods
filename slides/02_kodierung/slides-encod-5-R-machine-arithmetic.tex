\documentclass[11pt,compress,t,notes=noshow, xcolor=table]{beamer}
\input{../../style/preamble}

% latex-math includes as needed
\input{../../latex-math/basic-math}
\input{../../latex-math/basic-ml}

% Lecture title always has to be there
\title{Algorithms and Data Structures}

\begin{document}

\titlemeta{Encoding}{Peculiarities of machine arithmetic}{figure_man/simple_calc.png}
{
  \item Associative and distributive properties
  \item Order of addition
  \item Calculation of variance
}



\begin{vbframe}{Peculiarities of machine arithmetic}

Common arithmetic properties are no longer fulfilled.

\vspace*{0.1cm}

For simplicity, we use decimal representation with $m = 4$ and rounding.

\begin{itemize}
\item \textbf{Associative property:}
 \begin{eqnarray*}
 &a = 4, b = 5003, c = 5000 \quad \Rightarrow \quad& \\
 &a = 0.4 \cdot 10^1, b = 0.5003 \cdot 10^4, c = 0.5 \cdot 10^4&
 \end{eqnarray*}
 \begin{eqnarray*}
 (\tilde a + \tilde b) &=& 0.4 \cdot 10^1 + 0.5003 \cdot 10^4 = 0.5007 \cdot 10^4 \\
 (\tilde a + \tilde b) + \tilde c &=& 0.5007 \cdot 10^4 + 0.5 \cdot 10^4 = 1.0007 \cdot 10^4 \\
 &\approx& 0.1001 \cdot 10^5 = 10010
 \end{eqnarray*}
 \begin{eqnarray*}
 (\tilde b + \tilde c) &=& 0.5003 \cdot 10^4 + 0.5 \cdot 10^4 = 1.0003 \cdot 10^4 \\
 &\approx& 0.1000 \cdot 10^5 \\
 (\tilde b + \tilde c) + \tilde a &=& 0.1000 \cdot 10^5 + 0.4 \cdot 10^1 = 0.10004 \cdot 10^5 \\
 &\approx& 0.1000 \cdot 10^5 = 10000
 \end{eqnarray*}

 \framebreak

\item \textbf{Distributive property:}
 \begin{eqnarray*}
 2 \cdot (\tilde b - \tilde c) &=& 2 \cdot (0.5003 \cdot 10^4 - 0.5 \cdot 10^4) \\
 &=& 0.0006 \cdot 10^4 = 6
 \end{eqnarray*}
 \begin{eqnarray*}
 (2 \cdot \tilde b - 2 \cdot \tilde c) &=& 2 \cdot 0.5003 \cdot 10^4 - 2 \cdot 0.5 \cdot 10^4 \\
 &=& 1.0006 \cdot 10^4 - 1 \cdot 10^4 \\
 &\approx& 0.1001 \cdot 10^5 - 0.1 \cdot 10^5 = 0.0001 \cdot 10^5 = 10
 \end{eqnarray*}
\end{itemize}
Problem in the second example: catastrophic cancellation.
\end{vbframe}



\begin{vbframe}{Examples}
\footnotesize
\vspace{0.5cm}
\begin{verbatim}
1e16 - 1e16
## [1] 0
\end{verbatim}

\vspace{0.2cm}
\begin{verbatim}
(1e16 + 1) - 1e16
## [1] 0
\end{verbatim}

\vspace{0.2cm}
\begin{verbatim}
(1e16 + 2) - 1e16
## [1] 2
\end{verbatim}

\lz
\normalsize
$1e16$ cannot be represented exactly since it is larger than $2^{53}$, hence the distance is greater than 1.
\end{vbframe}



\begin{vbframe}{Examples}
\lz
\footnotesize
\begin{verbatim}
x = seq(1, 2e16, length = 100000)
s1 = sum(x)
s2 = sum(rev(x))
s1
## [1] 1e+21
\end{verbatim}

\vspace{0.3cm}
\begin{verbatim}
s2
## [1] 1e+21
## [1] 1e+21
\end{verbatim}

\vspace{0.3cm}
\begin{verbatim}
s1 - s2
## [1] -262144
\end{verbatim}

\end{vbframe}

\normalsize

\begin{vbframe}{Order of addition}

{\bf General recommendation:} Start with numbers having the smallest absolute values.

\lz

Assuming $0 \leq a_1 \leq a_2 \dots \leq a_n$, there are still various ways to perform the summation, e.g.:
\begin{itemize}
\item $ (( (a_1 + a_2) + a_3) + a_4) + a_5$
\item $ ((a_1 + a_2) + (a_3 + a_4) ) + a_5$
\item $ ((a_1 + a_2) + a_3) + (a_4 + a_5)$
\end{itemize}

\lz

Remark: Particularly bad errors can occur when calculating differences of numbers on computers (this will be discussed in another lecture). 

\end{vbframe}


\begin{vbframe}{Calculation of variances}
Sample: $x_1 = 356, x_2 = 357, x_3 = 358, x_4 = 359, x_5 = 360$
$$
4 S^2 = \sum\limits_{i=1}^5 (x_i - \bar x)^2 = \sum\limits_{i=1}^5 x_i^2 - 5(\bar x)^2 = 10
$$

Not like that in decimal machine arithmetic with $m=4$:

\lz 

First formula OK, but second one is a disaster:
\begin{eqnarray*}
&\xt_1^2 = .1267E6,\ \xt_2^2 = .1274E6,\ \xt_3^2 = .1282E6, & \\\
&\xt_4^2 = .1289E6,\ \xt_5^2 = .1296E6, & \\
&\sum \xt_i^2 = .6408E6 \qquad 5 \cdot (\bar {\xt})^2 = 5 \cdot .1282E6 = .6410E6&
\end{eqnarray*}

The second formula gives a negative empirical variance!


\framebreak

Three approaches to calculate the $1/n$ normalized standard deviation of a sample:
\lz
\footnotesize
\begin{verbatim}
sd1 = function(x) {
  s2 = mean((x - mean(x))^2)
  sqrt(s2)
}
\end{verbatim}

\vspace{0.1cm}
\begin{verbatim}
sd2 = function(x) {
  s2 = mean(x^2) - mean(x)^2
  sqrt(s2)
}
\end{verbatim}

\vspace{0.1cm}
\begin{verbatim}
sd3 = function(x) {
  n = length(x)
  s2 = ((n - 1) / n) * var(x)
  sqrt(s2)
}
\end{verbatim}


\framebreak
\lz

\begin{verbatim}
options("digits" = 20)
sd1(1:9)
## [1] 2.5819888974716112
\end{verbatim}

\vspace{0.3cm}
\begin{verbatim}
sd2(1:9)
## [1] 2.5819888974716116
\end{verbatim}

\vspace{0.3cm}
\begin{verbatim}
sd3(1:9)
## [1] 2.5819888974716112
\end{verbatim}


\normalsize
\framebreak

\begin{algorithm}[H]
  \begin{center}
  \caption{Calculation of variance in \texttt{R} (simplified)}
    \begin{algorithmic}[1]
    \State \textbf{Input:} $x \in \R^n$
    \State $s1 = s2 = 0;$
    \For{i = 1, ..., n}
      \State $s1 = s1 + x[i]$
    \EndFor
    \State $\text{xm} \leftarrow \frac{s1}{n}$
    \For{i = 1, ..., n}
      \State $s2 = s2 + (x[i] - xm) * (x[i] - xm)$
    \EndFor \\
    \Return $\frac{s2}{n - 1}$
    \end{algorithmic}
    \end{center}
\end{algorithm}

\vspace*{-0.5cm}

\citelink{GUPTA2024COVC}
% \begin{footnotesize}
% \url{https://github.com/SurajGupta/r-source/blob/56becd21c75d104bfec829f9c23baa2e144869a2/src/library/stats/src/cov.c}
% \end{footnotesize}

\end{vbframe}

% \begin{vbframe}{Algorithmus zu Beginn des Kapitels}
% \vspace*{-0.5cm}
% <<>>=
% exp2 = function(x) {
%   oldsum = 0
%   newsum = n = term = 1
%   while(oldsum != newsum){
%     oldsum = newsum
%     term = term * x / n
%     n = n + 1
%     newsum = oldsum + term
%   }
%   return(newsum)
% }
% exp2(-707)

% exp2(-708)
% @

% Nun auch klar warum \textbf{exp2} schlecht funktioniert für $x < 0$

% $\quad \Rightarrow \quad$ Differenz von sehr vielen betragsmäßig großen
% Zahlen

% $\quad \Rightarrow \quad$  Für $|x|$ sehr groß fast schon ein Zufallsergebnis

% Hier einfache Lösung des Problems  $\exp(-x) = 1 / \exp(x)$
% \end{vbframe}
\endlecture
\end{document}

% https://github.com/SurajGupta/r-source/blob/56becd21c75d104bfec829f9c23baa2e144869a2/src/library/stats/src/cov.c

