
%<<setup-child, include = FALSE>>=
%library(knitr)
%library(ggplot2)
%library(microbenchmark)
%set_parent("../style/preamble.Rnw")
%@

\input{../../2021/style/preamble4tex}
\input{../../latex-math/basic-math}

\begin{document}

\lecturechapter{6}{Introduction to Random Numbers}
\lecture{CIM1 Statistical Computation}

% \begin{vbframe}{Notation}
% 
% \begin{itemize}
% \item $x \text{ mod } m$: Modulo operation (divide by $m$ with remainder)
% \item $\oplus$: Bitwise XOR / addition modulo 2
% \item $(x_i)_{i \in \N}$: Sequence of machine numbers $x_i \in \M$
% \item $********_{16}$: Hexadecimal number
% \item $X$: Real random variable
% \item $f_X$: Density of $X$
% \item $F_X:= \P(X \le x)$: Distribution function of $X$
% \item $U \sim \text{U}(a, b)$: $U$ is uniformly distributed on the interval $(a, b)$
% \end{itemize}
% 
% 
% \end{vbframe}

\begin{vbframe}{True Random Numbers}
To source \enquote{true} random numbers there are several possibilities:
\begin{itemize}
  \item Tossing a coin, dice, roulette wheel, \ldots.
  \item Noise from electronic components (e.g. device drivers of a computer)
  \item Noise in the atmosphere (\url{http://random.org})
  \item Radioactive decay (example HRNG: \url{https://bit.ly/2NZ8whF})
  \item Response time of a user to a command prompt, time differences when typing on the keyboard (both measured in milliseconds or even more accurately),
    random mouse movement, \ldots
  \item \ldots
\end{itemize}
They all have in common that the creation of long sequences is very time-consuming or even impossible.\\
\medskip
Also, an exact repetition of a \enquote{random experiment} is not possible.

\framebreak

\begin{center}
\includegraphics[width =1\textwidth]{figure_man/random.png}
\end{center}

%<<eval=FALSE, include=FALSE>>=
%library("random")
%x = randomNumbers(5000, min = 1, max = 1e+06, col = 2) / 1e+06
%summary(x)
%@
%<<echo=FALSE,>>=
%x = load2("random.rda")
%# summary(x)
%@

%<<echo=FALSE>>=
%par(mfrow = c(1, 2))
%colnames(x) = c("x", "y")
%plot(x, pch = ".")
%hist(matrix(x, ncol = 1), freq = FALSE, main = "Histogram", xlab = "c(x,y)")
%@
\end{vbframe}

\begin{vbframe}{Pseudo-random Numbers}
\begin{itemize}
 %\item \enquote{Random} source of random numbers, through simple recursive algorithm on the computer.
 \item \textbf{Definition}: A sequence of pseudo-random numbers is a
  deterministic (!) sequence of numbers with the same relevant
  properties as a sequence of independent and identically distributed
  random variables.
 \item Usually numbers from discrete and continuous uniform distributions,
  other distributions result from transformations.
 \item Starting point: Let $x_i$ be a sequence of natural numbers with discrete uniform distribution on the
  interval $[0,m]$. Then $u_i=x_i/m$ (with $m$ being large) is an approximation of the continuous uniform distribution with numerical precision $1/m$.
\end{itemize}


\framebreak

\begin{description}
 \item[Initial value $x_0$:] Must be specified, initialization is frequently done based on the system time. \enquote{Random} experiments with
  \textbf{Pseudo-random number generators (PRNG)} can be reproduced completely
  if the initial value is known (\enquote{seed value}).
 \item[Period:] Since there is only a finite set of numbers available, at some point $x_k=x_0$ must hold and the sequence repeats itself: $x_{i+k}=x_i$, $x_{i+k+1}=x_{i+1}$, \ldots
\end{description}

\lz

PRNG with periods as long as possible are desirable.

\end{vbframe}


%\section{Uniform distribution}

\begin{vbframe}{Quality criteria for PRNG}
Relevant properties of PRNG for uniformly distributed random numbers on $(0, 1)$:
\begin{itemize}
\item Actual uniform distribution on $(0, 1)$.
\item Distributions of pairs, triplets, etc. are also random (especially for multidimensional
  uniform distribution).
\item Period length. If the period is too short, then \ldots
\begin{itemize}
\item \ldots many integers cannot be realized (problem due to uniform distribution)
\item \ldots there are not enough independent random numbers
\end{itemize}
\end{itemize}
The \emph{relevance of each property} depends very much on the respective
application.
\end{vbframe}

\begin{vbframe}{Testing PRNG}
In order to assess the quality of PRNGs, they are subject to strict random number test suites.
The best known is \enquote{Die Hard} from George
Marsaglia (available for free):
\begin{itemize}
 \item Considered individually, is each bit i.i.d.\ 0 or 1 with probability $1/2$?
 \item Spectral test: Are $n$-tuples uniformly distributed in the unit cube?
  Are there autocorrelations?
 \item Rank of binary $6\times 8$ and $32\times 32$ matrices.
 \item \proglang{R} package \pkg{RDieHarder} \\
    \url{http://cran.r-project.org/web/packages/RDieHarder/}
\end{itemize}
\end{vbframe}


\endlecture

\end{document}