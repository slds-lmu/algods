\documentclass[11pt,compress,t,notes=noshow, xcolor=table]{beamer}
\input{../../style/preamble}

% latex-math includes as needed
\input{../../latex-math/basic-math}
\input{../../latex-math/basic-ml}

% Lecture title always has to be there
\title{Algorithms and Data Structures}

\begin{document}

\titlemeta{Random Numbers}{Mersenne Twister \& \texttt{R}}
{figure_man/mersenneinit.png}
{
  \item Mersenne Twister algorithm
  \item Properties of Mersenne Twister
  \item Implementation in \texttt{R}
}

\begin{vbframe}{Mersenne Twister}

The \textbf{Mersenne Twister} is currently the most frequently used random number generator and was developed in 1997 by M. Matsumoto and T. Nishimura.

\begin{center}
\includegraphics[width = 0.6\textwidth]{figure_man/mersenne.png} \\
\begin{footnotesize}
\url{https://www.cryptologie.net/article/331/how-does-the-mersennes-twister-work/}
\end{footnotesize}
\end{center}


\framebreak

\begin{small}
	\textbf{Note:} Here, random numbers $x_i$ are represented by $w$-bit vectors (usually $w = 32, 64$). To emphasize this, we write $\mathbf{x}_i$ (in bold).
\end{small}

\vspace*{0.2cm}

\textbf{Description of the algorithm:}
\begin{enumerate}
\item \textbf{Initialization: } A seed $\mathbf{x}_0$ is set, and the first $n$ values are calculated based on it (not described here). These values are not part of the final output.
\vspace*{0.2cm}
\begin{center}
\includegraphics[width = 0.6\textwidth]{figure_man/mersenneinit.png}
\end{center}

\framebreak

\item \textbf{Recursion:} Formally the following recursion formula is used

$$
\mathbf{x}_{k + n} = \mathbf{x}_{k + m}\ \underbrace{\bigoplus}_{3.}\ \underbrace{(\mathbf{x}_{k}^l ||  \mathbf{x}_{k + 1}^r)}_{1.}\ \underbrace{\mathbf{A}}_{2.}
$$

\begin{itemize}
	\item $n$: Degree of recurrence, "size" of blocks
	\item $m$: Integer $1 \le m < n$
	\item $\mathbf{x}_{k}^l$, $\mathbf{x}_{k}^r$: Left and right part of the vector $x_{k}$
	\item $0 \le c \le w - $1 determines where the "left part" ends
\end{itemize}

\vspace*{0.2cm}

\begin{center}
	\includegraphics[width = 0.5 \textwidth]{figure_man/twist.png}
\end{center}

\framebreak

For ease of exposition, let $w = 4$, $c = 2$.

\lz

\textbf{1. Concatenation:}

$$
\begin{array}{r} \mathbf{x}_k \\ \mathbf{x}_{k + 1} \\ \mathbf{x}_{k + 2} \end{array}\begin{pmatrix} \textcolor{green}{0} & \textcolor{green}{1} & 1 & 0 \\
\textcolor{red}{1} & \textcolor{red}{0} & \textcolor{green}{1} & \textcolor{green}{0} \\
1 & 1 & \textcolor{red}{1} & \textcolor{red}{1}
\end{pmatrix} \Rightarrow \begin{matrix}
(x_{k}^l || x_{k + 1}^r) & = (\textcolor{green}{0}, \textcolor{green}{1}, \textcolor{green}{1}, \textcolor{green}{0}) \\
(x_{k + 1}^l || x_{k + 2}^r) & = (\textcolor{red}{1}, \textcolor{red}{0}, \textcolor{red}{1}, \textcolor{red}{1})
\end{matrix}
$$

\framebreak
\begin{footnotesize}
\textbf{2. Multiplication by $\mathbf{A}$: }


We multiply with the so-called \textbf{Twist Matrix} $\mathbf{A}$

$$
\mathbf{A} = \begin{pmatrix}
0 & 1 & 0 & 0  \\
0 & 0 & 1 & 0 \\
 0 & 0 & 0 & 1 \\
a_{3} & a_{2} & a_{1} & a_0
\end{pmatrix}
$$

\begin{eqnarray*}
(0, 1, 1, 0)  \mathbf{A} &=& (0, 0, 1, 1)\\
(1, 0, 1, 1) \mathbf{A} &=& (0 \oplus a_3, 1 \oplus + a_2, 0 \oplus a_1, 1 \oplus a_0)
\end{eqnarray*}

In summary $\quad
\mathbf{x}\mathbf{A} = \begin{cases}
\text{shift}(\mathbf{x}), \quad \text{if last bit } x_0 = 0 \\
\text{shift}(\mathbf{x}) \oplus \mathbf{a} , \quad \text{if } x_0 = 1
\end{cases}
$

%\framebreak
\lz

\textbf{3. XOR: }


In the last step a bitwise XOR is calculated, e.g.

$$
\mathbf{x}_{k + m} \bigoplus (0, 0, 1, 1)
$$
\end{footnotesize}

\framebreak

\item \textbf{Tempering:}  In the last step, tempering is applied to the generated random numbers in order to improve their distribution properties.

\vspace*{0.2cm}

\begin{center}
	\includegraphics[width = 0.3 \textwidth]{figure_man/temper.png}
\end{center}
\end{enumerate}

\framebreak

The following operations are performed:
\vspace*{-0.2cm}
\begin{eqnarray*}
\mathbf{x} &\mapsto& \mathbf{y} := \mathbf{x} \oplus ((\mathbf{x} >> u) \text{ AND } \mathbf{d}) \\ \mathbf{y} &\mapsto& \mathbf{y} := \mathbf{y} \oplus ((\mathbf{y} << s) \text{ AND } \mathbf{b}) \\
\mathbf{y} &\mapsto& \mathbf{y} := \mathbf{y} \oplus ((\mathbf{y} < < t) \text{ AND } \mathbf{c}) \\ \mathbf{y} &\mapsto& \mathbf{z} := \mathbf{y} \oplus (\mathbf{y} >> l)
\end{eqnarray*}

where $\mathbf{x} >> u$ ($\mathbf{x} << u$) describes the bitwise "right"-shift ("left"-shift) by $u$ and "AND" describes the bitwise "and".

\lz

This can be summarized as follows:

$$\mathbf{x} \mapsto \mathbf{z} = \mathbf{x}\mathbf{T}. $$


\framebreak

Coefficients for MT19937: (standard implementation 32-bit)

\begin{description}
  \item[w] word size (in number of bits): 32
  \item[n] degree of recurrence: 624
  \item[m] middle word, an offset used in the recurrence relation defining the series x: 397
  \item[c] separation point of one word, or the number of bits of the lower bitmask: 31
  \item[a] coefficients of the rational normal form twist matrix: $9908B0DF_{16}$
  \item[u, d, l] tempering masks/shifts: $(11, FFFFFFFF_{16}, 18)$
  \item[s, b] tempering masks/shifts: $(7, 9D2C5680_{16})$
  \item[t, c] tempering masks/shifts:  $(15, EFC60000_{16})$
\end{description}

\framebreak


\textbf{Properties}:

\begin{itemize}
	\item Extremely long period of $2^{19937} - 1\approx 4.3 \cdot 10^{6001}$ (so-called "Mersenne prime")
	\item All bits of the output sequence are uniformly distributed $\to$ thus the corresponding integer values are also uniformly distributed
	\item Low correlation of consecutive values
	\item Fast implementation by calculating $n$ ($n = 624$ in MT19937) random numbers in one step
	\item Highly parallelizable
\end{itemize}

\lz

\footnotesize{
Further information:
\begin{itemize}
	\item \url{www.math.sci.hiroshima-u.ac.jp/~m-mat/MT/ARTICLES/mt.pdf} (Original paper Mersenne Twister)
	\item \url{http://statweb.stanford.edu/~owen/mc/Ch-unifrng.pdf} (Lecture on PRNGs)
\end{itemize}
}
% https://bugs.r-project.org/bugzilla/show_bug.cgi?id=17494

\end{vbframe}


% \subsection{Shift Register Methoden}


% \begin{vbframe}{Twisting / Shuffling}
% Multiplikative und lineare Kongruenzgeneratoren haben zu hohe
% Autokorrelation, insbesondere sind $n$-Tupel nicht gleichverteilt im
% $n$-dimensionalen Einheitswürfel, sondern konzentrieren sich nahe
% weniger Hyperebenen.\\
% \medskip
% Eine einfache Möglichkeit, um zumindest Tupel der Länge $n$ ohne
% Autokorrelation zu bekommen, ist diese automatisch zu mischen.
% \begin{enumerate}
%  \item Initialisiere $s_i=u_i$, $i=1,\ldots,n$, und $y=s_n$.
%  \item Erzeuge neuen Wert $u$, setze $j=\lceil y \cdot n \rceil$.
%  \item Setze $y=s_j$ und $s_j=u$.
%  \item Gib Zufallszahl $y$ zurück, wiederhole ab 2.
% \end{enumerate}
% $u_i$ aus einem Standardgenerator, Seed sind die $n$ Zahlen $s_n$.
%
% \framebreak
%
% <<echo=FALSE>>=
% shuffle = function(n, rand = mkg, ...) {
%   u = mkg(n * 2, ...)
%   s = u[1:n]
%   k = n
%   x = NULL
%   for(i in 1:n) {
%     j = as.integer(s[k] * k) + 1
%     x = c(x, s[k])
%     s[k] = s[j]
%     s[j] = u[n + i - 1]
%   }
%   return(x)
% }
% @
%
% <<echo=FALSE>>=
% randu = shuffle(12000, 1, m = 2^31, a = 2^16 + 3)
% randu3D = as.data.frame(matrix(randu, ncol = 3, byrow = TRUE))
% colnames(randu3D) = c("x1", "x2", "x3")
% @
%
% <<echo=FALSE, rgl=TRUE>>=
% library("rgl")
% mfrow3d(1, 3)
% with(randu3D, plot3d(x1, x2, x3))
% rgl.viewpoint(theta = -12.8, phi = 3.3, fov = 0, zoom = 0.7)
% with(randu3D, plot3d(x1, x2, x3))
% rgl.viewpoint(theta = -6.8, phi = 3.3, fov = 0, zoom = 0.7)
% with(randu3D, plot3d(x1, x2, x3))
% rgl.viewpoint(theta = -3.8, phi = 3.8, fov = 0, zoom = 0.7)
% @
%
% {\footnotesize \emph{Hinweis:} Erneut der selbe Plot
% aus drei verschiedenen Perspektiven.}
%
% % <<>>=
% % z = with(randu, (6 * x2 - 9 * x1) %% 1)
% % head(cbind(z, randu$x3))
% % @
% \end{vbframe}
%
%
% \begin{fragileframe}{Shift-Register-Methode}
% \textbf{Grundlage:} Bei der Shift-Register-Methode wird eine Bitfolge X
% durch Rekursion erzeugt. Für das Element n der Folge gilt:
% $$
% X_n = X_{n-p} \oplus  X_{n-p + q}, \quad 1 < q < p,
% $$
% wobei $\oplus$ für exklusives Oder steht:
%
% \lz
%
% \begin{table}
% \centering
% \begin{tabular}{|cc|c|}
%   \hline
% $X_{n-p}$ & $X_{n-p + q}$  & $X_n$ \\
%   \hline
% \textcolor{green}{0} & \textcolor{green}{0} & \textcolor{red}{0} \\
% \textcolor{green}{0} & \textcolor{green}{1} & \textcolor{red}{1} \\
% \textcolor{green}{1} & \textcolor{green}{0} & \textcolor{red}{1} \\
% \textcolor{green}{1} & \textcolor{green}{1} & \textcolor{red}{0} \\
%    \hline
% \end{tabular}
% \end{table}
%
% Es wird eine Startfolge der Länge p benötigt.
%
%
% <<include=FALSE>>=
% source("rsrc/ieee754.R")
% as.dec.bin("1011")
% as.bin(11)
% as.dec(as.bin(11))
% @
%
% \framebreak
%
% <<echo=FALSE>>=
% shift = function(n, seed = 11, bits = 4, p = 4, q = 3) {
%   x = as.bin(seed, unsigned = TRUE)
%   x = as.integer(strsplit(x, "")[[1]])
%   k = length(x)
%   for(i in 1:(n * bits)) {
%     x = c(x, as.integer(xor(x[k - p + 1], x[k - p + q + 1])))
%     k = length(x)
%   }
%   i = split(1:k, ceiling(seq_along(1:k) / bits))
%   x = as.character(sapply(i, function(j) {
%     paste(x[j], collapse = "")
%   }))
%   x = sapply(x, function(b) {
%     if(nchar(b) < bits) {
%       b = paste(b, paste(rep(0, bits - nchar(b)),
%         collapse = ""),
%       sep = "")
%     }
%     b
%   })
%   ix = sapply(x, as.dec.bin)
%   rval = data.frame(
%     "bin" = x,
%     "int" = ix,
%     "u" = ix / (2^bits),
%     stringsAsFactors = FALSE
%   )
%   rval
% }
% @
%
% % <<eval=FALSE>>=
% % shift = function(n, seed = 11, bits = 4, p = 4, q = 3) {
% %   x = as.bin(seed, unsigned = TRUE)
% %   x = as.integer(strsplit(x, "")[[1]])
% %   k = length(x)
% %   for(i in 1:(n * bits)) {
% %     x = c(x, as.integer(xor(x[k - p + 1], x[k - p + q + 1])))
% %     k = length(x)
% %   }
% %   i = split(1:k, ceiling(seq_along(1:k) / bits))
% %   x = as.character(sapply(i, function(j) {
% %     paste(x[j], collapse = "")
% %   }))
% % @
%
% % <<eval=FALSE>>=
% %   x = sapply(x, function(b) {
% %     if(nchar(b) < bits) {
% %       b = paste(b, paste(rep(0, bits - nchar(b)),
% %         collapse = ""),
% %       sep = "")
% %     }
% %     b
% %   })
% %   ix = sapply(x, as.dec.bin)
% %   rval = data.frame(
% %     "bin" = x,
% %     "int" = ix,
% %     "u" = ix / (2^bits),
% %     stringsAsFactors = FALSE
% %   )
% %   rval
% % }
% % @
%
%
% \textbf{Beispiel:} $p = 4, q = 3$, Startfolge: $1011$.
%
% \begin{footnotesize}
%
% \begin{eqnarray*}
% X_5 &=& X_{1} \oplus  X_{4} = 1 \oplus 1 = 0 \\
% X_6 &=& X_{2} \oplus  X_{5} = 0 \oplus 0 = 0 \\
% X_7 &=& X_{3} \oplus  X_{6} = 1 \oplus 0 = 1 \\
%     &\hdots&
% \end{eqnarray*}
%
% \begin{table}
% \begin{tabular}{rrrrrrrrrr}
%   \hline
%   &   &   & \textcolor{green}{1} & 0 & 1 & 1 & & & \\
%   \hline
% 1 & 0 & 1 & \textcolor{green}{1} & \textcolor{red}{0} & & & & \\
%   \hline
%   \hline
%   &   &   & 1 & \textcolor{green}{0} & 1 & 1 & 0 & &\\
%   \hline
% 1 & 0 & 1 & 1 & \textcolor{green}{0} & \textcolor{red}{0} & & & &\\
%   \hline
%   \hline
%   &   &   & 1 & 0 & \textcolor{green}{1} & 1 & 0 & 0 &\\
%   \hline
% 1 & 0 & 1 & 1 & 0 & \textcolor{green}{0} & \textcolor{red}{1} & & &\\
%   \hline
%   \hline
%   &   &   & 1 & 0 & 1 & \textcolor{green}{1} & 0 & 0 & 1\\
%   \hline
% 1 & 0 & 1 & 1 & 0 & 0 & \textcolor{green}{1} & \textcolor{red}{0} & & \\
%   \hline
% \end{tabular}
% \end{table}
%
% \end{footnotesize}
% \framebreak
%
% <<size="scriptsize">>=
% shift(15)
% @
%
% \framebreak
%
% Im Beispiel codieren Vierergruppen Integer zwischen 1 und 15
% $$
% \text{Zufallszahlen} \quad y_i = \sum\limits_{j=1}^4 2^{-j} X_{4i+j}
% $$
% Periode der Länge $15 = 2^4 - 1$
% (gilt auch für Bits).
% Im Allgemeinen muss nicht gelten, dass Gruppenlänge $M = p$.\\
% \bigskip
% Wahl der Parameter nicht egal. Wähle oben $q = 2$:
%
% <<>>=
% shift(3, q = 2)
% @
%
% \framebreak
%
% Für $q = 1$ und $p = 5$ und Bitlänge 5, erhält man auch nur Periode der Länge 3:
%
% <<>>=
% shift(9, seed = 22, q = 1, p = 5, bits = 5)
% @
% \end{fragileframe}
%
%
% \begin{vbframe}{Generalized Feedback Shift Register (GFSR)}
% Lewis und Payne (1973):
% $$
% y_i = \sum\limits_{j=1}^M 2^{-j} X_{i-s(j)}
% $$
% Typisch: $s(1)=0 < s(2) < \dots < s(M)$ \\
% Ursprünglicher Vorschlag: $s(j) = 100 p (j-1)$, mit $p = 98$ und $q = 27$.
% \begin{itemize}
% \item Liefert große Integer mit langen Perioden $(2^p-1)$.
% \item unabhängig von Maschinenarithmetik (brauche nur $\oplus$)
% \item hervorragende multivariate Eigenschaften
% \end{itemize}
% {\bf Größter Nachteil:}  Nicht triviale Initialisierung. \\
% Benötige Startsequenz der Länge $p$ und nicht jede Startsequenz garantiert optimale Eigenschaften (vgl. Monahan).
% \end{vbframe}
%
%
% \begin{vbframe}{Mersenne-Twister}
% Bei GSFR gilt Rekurrenzrelation für alle Komponenten der Vektoren\\
% $Y_n = [X_n, X_{n - s(2)}, \dots,  X_{n - s(m)}]$.
%
% \lz
%
% {\bf Twisted GFSR}: Variation der Rekursion zu
% $$
% Y_{n+p} = Y_n\ \bigoplus\ (Y_{n+q}^l, Y_{n+q+1}^r)\ \mathbf{A},
% $$
% wobei $Y_{n}^l$ die erste Hälfte und $Y_{n}^r$ die zweite Hälfte
% der Bits und $\mathbf{A}$ eine geeignete Matrix
%
% \lz
%
% {\bf Mersenne-Twister}: Matsumoto und Nishimura (1998),\\
% $p = 624$, $q = 397$, Matrix $\mathbf{A}$ erzeugt nur einfache Bit-Shifts.
% \end{vbframe}


\begin{vbframe}{Transformation to the Interval $(0, 1)$}

\vspace*{-0.4cm}

\begin{itemize}
\item In the procedures discussed, integers between $0$ and $m - 1$ were generated.
\item If real random numbers in the interval $(0, 1)$ need to be simulated, a simple division by $m$ is sufficient.
\item \textbf{Problem:} The random number can be $0$, while for an "actual" $(0, 1)$ - uniformly distributed random variable $U$ the following holds:
$$
\mathbb{P}(U = 0) = 0
$$
\item This is often a problem in practical applications, e.g. with $\log(U)$.
\item If a $0$ is generated, the random number can be discarded or an error message is issued.
\item Since modern congruential generators have a long period, this is unlikely.
\end{itemize}
%
% \begin{table}
% \begin{footnotesize}
% \centering
% \begin{tabular}{r|rrrrrrrrrr}
%   \hline
%  i & 1 & 2 & 3 & 4 & 5 & 6 & 7 & 8 & 9 & 10 \\
%   \hline
% 0  & 0.29 & 0.94 & 0.41 & 0.12 & 0.18 & 0.76 & 0.65 & 0.47 & 0.71 & 0.06 \\
% 10 & 0.59 & 0.88 & 0.82 & 0.24 & 0.35 & 0.53 & 0.29 & 0.94 & 0.41 & 0.12 \\
% 20 & 0.18 & 0.76 & 0.65 & 0.47 & 0.71 & 0.06 & 0.59 & 0.88 & 0.82 & 0.24 \\
% 30 & 0.35 & 0.53 & 0.29 & 0.94 & 0.41 & 0.12 & 0.18 & 0.76 & 0.65 & \ldots \\
%    \hline
% \end{tabular}
% \end{footnotesize}
% \end{table}

\end{vbframe}

\begin{vbframe}{PRNGs in \texttt{R}}
The Mersenne Twister is (currently) the default method in \texttt{R} with period
$2^{19937} - 1 \approx 10^{6001}$ and guaranteed uniform distribution in
623 dimensions.
Seeds are 624 32-bit integers on top of the current position in this set. The \code{set.seed()} function generates a valid seed from a
single integer value using the linear congruential generator with
$$
  m=2^{32}, \qquad a=69069, \qquad b=1
$$
There are a number of other generators available. Furthermore, the
user can also specify her own generator as default.
\code{RNGversion('x.y.z')} can be used to set the random generators as they were in an earlier \texttt{R} version (for reproducibility) (Wichman-Hill up to 0.98, Marsaglia-Multicarry up to 0.00).
1.6.1). Initialization of the seed via time.
\end{vbframe}


\begin{vbframe}{R: Random Number Generation}
\begin{itemize}
%\begin{footnotesize}
\item \code{set.seed()} function

\footnotesize
\begin{verbatim}
set.seed(123); u1 <- runif(100)
set.seed(123); u2 <- runif(100) 
identical(u1, u2) # the same because of identical RNG status
## [1] TRUE
\end{verbatim}


\normalsize
\item \code{.Random.seed()} is an integer vector, containing the random number generator (RNG) state for random number generation in \texttt{R}. It can be saved and restored, but should not be altered by the user.

\item \code{RNGkind()} is a more friendly interface to query or set the kind of RNG in use.
\footnotesize

\begin{verbatim}
# default for "kind", "normal kind" and "sample kind" 
RNGkind("default") 
RNGkind()
## [1] "Mersenne-Twister" "Inversion"  "Rejection"
.Random.seed[1:3] # the default random seed is 626 integers
## [1] 10403 624 1858651209
\end{verbatim}


\normalsize
\framebreak
\item Change default of kind ("Mersenne-Twister") to "Wichmann-Hill"

\footnotesize
\begin{verbatim}
RNGkind("Wich")
RNGkind()
## [1] "Wichmann-Hill" "Inversion" "Rejection"
.Random.seed
## [1] 10400  5989  8337  9843
\end{verbatim}


\normalsize
\item Change methods depending on defaults in a specific \texttt{R} Version

\footnotesize
\begin{verbatim}
RNGversion(getRversion()) # current version
RNGkind()
## [1] "Mersenne-Twister" "Inversion" "Rejection"
RNGversion("1.0.0") # first \texttt{R} version
## Warning in RNGkind("Marsaglia-Multicarry", "Buggy
Kinderman-Ramage", "Rounding"): buggy version of 
Kinderman-Ramage generator used
## Warning in RNGkind("Marsaglia-Multicarry", "Buggy
Kinderman-Ramage", "Rounding"): non-uniform 'Rounding' 
sampler used
\end{verbatim}


%\end{footnotesize}
\end{itemize}
\end{vbframe}





\begin{vbframe}{Parallel computing}
A complex topic is the application of random number generators for parallel computing, where a long calculation is split between several machines and processed in parallel.
Usually, a "master" distributes the jobs to several "slaves".
Initialization of seed using the two "standard" methods
\begin{itemize}
  \item Time of day or
  \item Fixed given number
\end{itemize}
is not useful.
Special algorithms for this purpose are provided e.g. in the \texttt{R} packages \pkg{rlecuyer} or
\pkg{rstreams} (both use the same algorithm from L'Ecuyer).
\end{vbframe}



\endlecture


\end{document}