\documentclass[11pt,compress,t,notes=noshow, xcolor=table]{beamer}
\input{../../style/preamble}

% latex-math includes as needed
\input{../../latex-math/basic-math}
\input{../../latex-math/basic-ml}

% Lecture title always has to be there
\title{Algorithms and Data Structures}

\begin{document}

\titlemeta{Matrix Decomposition}{QR Decomposition}
{figure_man/frame_118_delay-4s.png}
{
  \item QR decomposition
  \item Gram-Schmidt Pprocess
}

\begin{vbframe}{QR decomposition}

% Als Alternative zu Cholesky wird mitunter die QR-Zerlegung angewandt.

Given $\Amat \in \R^{n \times n}$. We decompose $\Amat$ into the product of an orthogonal matrix $\bm{Q}\in \R^{n \times n}$ and an upper triangular matrix $\bm{R} \in \R^{n \times n}$

$$
\Amat = \bm{QR} \quad \mbox{with} \quad \bm{Q}^\top\bm{Q} = \bm{I},
$$

The columns of the matrix $\bm{Q} = (\bm{q}_1, \ldots, \bm{q}_n)$ form an orthonormal basis for the column space of the matrix $\Amat$ and

$$
\bm{R} = \mat{
\langle \bm{q_1}, \bm{a_1} \rangle & \langle \bm{q_1}, \bm{a_2} \rangle & \langle \bm{q_1}, \bm{a_3} \rangle & \cdots \\
0                                  & \langle \bm{q_2}, \bm{a_2} \rangle & \langle \bm{q_2}, \bm{a_3} \rangle & \cdots \\
0                                  & 0                                  & \langle \bm{q_3}, \bm{a_3} \rangle & \cdots \\
\vdots                             & \vdots                             & \vdots                             & \ddots \\}
$$

The orthonormal basis for $\Amat$ is calculated by the Gram-Schmidt process.

\end{vbframe}

\begin{vbframe}{Gram-Schmidt process}

The process takes a finite, linearly independent set of vectors and generates an orthogonal set of vectors that form an orthonormal basis.$^{(*)}$

\vspace*{0.2cm}
\begin{footnotesize}
\textbf{Procedure:}
Projection: $\text{proj}_{\bm{q}} \bm{a} = \frac{\langle \bm{q}, \bm{a} \rangle}{\langle \bm{q}, \bm{q} \rangle} \bm{q}$.
\vspace*{-0.1cm}

\begin{align*}
\bm{u}_1 & = \bm{a}_1                                     & \bm{q}_1 & = \frac{\bm{u}_1}{\|\bm{u}_1\|}\\
\bm{u}_2 & = \bm{a}_2 - \text{proj}_{\bm{u}_1} \bm{a}_2   & \bm{q}_2 & = \frac{\bm{u}_2}{\|\bm{u}_2\|}\\
\vdots   & = \vdots                                       & \vdots   & = \vdots \\
\bm{u}_k & = \bm{a}_k - \sum_{j=1}^{k-1} \text{proj}_{\bm{u}_j} \bm{a}_k & \bm{q}_k & = \frac{\bm{u}_k}{\|\bm{u}_k\|}
\end{align*}
\end{footnotesize}
The vectors constructed in this way actually form an orthonormal basis of the column space of $\Amat$ (can be shown).
\vfill

\begin{footnotesize}
$^{(*)}$ If the vector $\bm{a}_i$ is not independent of $\bm{a}_1, ..., \bm{a}_{i - 1}$, then $\bm{u}_i = \bm{0}$. %The $i$th calculation step is skipped.
\end{footnotesize}

% \framebreak
% 
% The vectors constructed in this way actually form an orthonormal basis of the column space of $\Amat$ (can be shown).


\framebreak

$\bm{A}$ can now be represented by the calculated orthonormal basis:

\begin{align*}
  \bm{a}_1 & = \bm{q}_1 \langle \bm{q}_1, \bm{a}_1 \rangle \\
  \bm{a}_2 & = \bm{q}_1 \langle \bm{q}_1, \bm{a}_2 \rangle + \bm{q}_2 \langle \bm{q}_2, \bm{a}_2 \rangle \\
  \vdots   & = \vdots \\
  \bm{a}_k & = \sum_{j=1}^{k} \bm{q}_j \langle \bm{q}_j, \bm{a}_k \rangle
\end{align*}

Or in matrix notation:

$$
  \bm{Q} \bm{R} = (\bm{q}_1 \langle \bm{q}_1, \bm{a}_1 \rangle,
                   \bm{q}_1 \langle \bm{q}_1, \bm{a}_2 \rangle + \bm{q}_2 \langle \bm{q}_2, \bm{a}_2 \rangle,
                   \cdots) = \bm{A}
$$

% \vspace*{0.2cm}
%
% \begin{footnotesize}
% \textbf{Beweis}:
%
% Wir zeigen die Orthogonalität der Vektoren $\bm{u}_1, ..., \bm{u}_l$ mittels Induktion:
%
% \begin{enumerate}
% \item $l = 1$: $\bm{u}_1 = \bm{a}_1$ ist (trivialerweise) linear unabhängig. \checkmark
% \item Induktionsschritt $l \to l + 1$: Es sei $j \le l$.
% \vspace*{-0.5cm}
%
% \begin{eqnarray*}
%   \scp{\bm{u}_{l + 1}}{\bm{u}_j} &=& \scp{\bm{a}_{l + 1}}{\bm{u}_j} - \sum_{k \le l} \frac{\scp{\bm{a}_{l + 1}}{\bm{u}_k}}{\scp{\bm{u}_k}{\bm{u}_k}} \underbrace{\scp{\bm{u}_k}{\bm{u}_j}}_{\substack{= 0 \text{ für } k \ne j \\ \text{Induktions-} \\ \text{ voraussetzung}}} \\
%   &=&  \scp{\bm{a}_{l + 1}}{\bm{u}_j} - \frac{\scp{\bm{a}_{l + 1}}{\bm{u}_j}}{\scp{\bm{u}_j}{\bm{u}_j}}\scp{\bm{u}_j}{\bm{u}_j} = 0
% \end{eqnarray*}
%
% Daher sind $\{\bm{u}_1, ..., \bm{u}_l, \bm{u}_{l + 1}\}$ und somit auch $\{\bm{q}_1, ..., \bm{q}_l, \bm{q}_{l + 1}\}$ linear unabhängig. \checkmark
%
% \end{enumerate}
%
% Da gezeigt wurde, dass die Spaltenvektoren von $\Amat$ über die Vektoren  $\bm{q}_1, ..., \bm{q}_n$ dargestellt werden können (s. vorhergehende Slide), folgt, dass $\bm{q}_1, ..., \bm{q}_n$ den Spaltenraum von $\Amat$ aufspannen. \checkmark
% \end{footnotesize}


\end{vbframe}

\begin{vbframe}{Gram-Schmidt visualized}

\begin{scriptsize}

\textbf{Given:} Three independent vectors \textcolor{red}{$a_1$}, \textcolor{green}{$a_2$}, \textcolor{blue}{$a_3$} \\
\textbf{Aim:} Vectors of an orthonormal basis \textcolor{red}{$q_1$}, \textcolor{green}{$q_2$}, \textcolor{blue}{$q_3$}

\setbeamercovered{transparent}
\begin{enumerate}
  \item \textcolor{red}{$a_1$} serves as the first vector of the orthogonal basis (\textcolor{red}{$u_1$}).
  \item<2-> \textcolor{green}{$a_2$} is projected onto \textcolor{red}{$u_1$}; projection is substracted from \textcolor{green}{$a_2$}
        to obtain \textcolor{green}{$u_2$}.
  \item<3-> \textcolor{blue}{$a_3$} is projected onto \textcolor{red}{$u_1$} and \textcolor{green}{$u_2$},
        to obtain \textcolor{blue}{$u_3$}.
  \item<4-> \textcolor{red}{$u_1$}, \textcolor{green}{$u_2$} and \textcolor{blue}{$u_3$} are normalized.
\end{enumerate}

\end{scriptsize}

\begin{figure}
  \centering
  \includegraphics[width=0.45\textwidth]{figure_man/frame_010_delay-2s.png}
  \includegraphics[width=0.45\textwidth]{figure_man/frame_013_delay-1s.png}
\end{figure}
\tiny{\url{https://commons.wikimedia.org/wiki/File:Gram-Schmidt_orthonormalization_process.gif}}

\framebreak

\begin{scriptsize}

\textbf{Given:} Three independent vectors \textcolor{red}{$a_1$}, \textcolor{green}{$a_2$}, \textcolor{blue}{$a_3$} \\
\textbf{Aim:} Vectors of an orthonormal basis \textcolor{red}{$q_1$}, \textcolor{green}{$q_2$}, \textcolor{blue}{$q_3$}

\setbeamercovered{transparent}
\begin{enumerate}
  \item<2-> \textcolor{red}{$a_1$} serves as the first vector of the orthogonal basis (\textcolor{red}{$u_1$}).
  \item \textcolor{green}{$a_2$} is projected onto \textcolor{red}{$u_1$}; projection is substracted from \textcolor{green}{$a_2$}
        to obtain \textcolor{green}{$u_2$}.
  \item<3-> \textcolor{blue}{$a_3$} is projected onto \textcolor{red}{$u_1$} and \textcolor{green}{$u_2$},
        to obtain \textcolor{blue}{$u_3$}.
  \item<4-> \textcolor{red}{$u_1$}, \textcolor{green}{$u_2$} and \textcolor{blue}{$u_3$} are normalized.
\end{enumerate}

\end{scriptsize}

\begin{figure}
  \centering
  \includegraphics[width=0.33\textwidth]{figure_man/frame_016_delay-3s.png}
  \includegraphics[width=0.33\textwidth]{figure_man/frame_019_delay-3s.png}
  \includegraphics[width=0.33\textwidth]{figure_man/frame_039_delay-4s.png}
\end{figure}
\tiny{\url{https://commons.wikimedia.org/wiki/File:Gram-Schmidt_orthonormalization_process.gif}}

\framebreak

\begin{scriptsize}

\textbf{Given:} Three independent vectors \textcolor{red}{$a_1$}, \textcolor{green}{$a_2$}, \textcolor{blue}{$a_3$} \\
\textbf{Aim:} Vectors of an orthonormal basis \textcolor{red}{$q_1$}, \textcolor{green}{$q_2$}, \textcolor{blue}{$q_3$}

\setbeamercovered{transparent}
\begin{enumerate}
  \item<2-> \textcolor{red}{$a_1$} serves as the first vector of the orthogonal basis (\textcolor{red}{$u_1$}).
  \item<3-> \textcolor{green}{$a_2$} is projected onto \textcolor{red}{$u_1$}; projection is substracted from \textcolor{green}{$a_2$}
        to obtain \textcolor{green}{$u_2$}.
  \item \textcolor{blue}{$a_3$} is projected onto \textcolor{red}{$u_1$} and \textcolor{green}{$u_2$},
        to obtain \textcolor{blue}{$u_3$}.
  \item<4-> \textcolor{red}{$u_1$}, \textcolor{green}{$u_2$} and \textcolor{blue}{$u_3$} are normalized.
\end{enumerate}

\end{scriptsize}

\begin{figure}
  \centering
  \includegraphics[width=0.45\textwidth]{figure_man/frame_060_delay-2s.png}
  \includegraphics[width=0.45\textwidth]{figure_man/frame_063_delay-1s.png}
\end{figure}
\tiny{\url{https://commons.wikimedia.org/wiki/File:Gram-Schmidt_orthonormalization_process.gif}}

\framebreak

\begin{scriptsize}

\textbf{Given:} Three independent vectors \textcolor{red}{$a_1$}, \textcolor{green}{$a_2$}, \textcolor{blue}{$a_3$} \\
\textbf{Aim:} Vectors of an orthonormal basis \textcolor{red}{$q_1$}, \textcolor{green}{$q_2$}, \textcolor{blue}{$q_3$}

\setbeamercovered{transparent}
\begin{enumerate}
  \item<2-> \textcolor{red}{$a_1$} serves as the first vector of the orthogonal basis (\textcolor{red}{$u_1$}).
  \item<3-> \textcolor{green}{$a_2$} is projected onto \textcolor{red}{$u_1$}; projection is substracted from \textcolor{green}{$a_2$}
        to obtain \textcolor{green}{$u_2$}.
  \item \textcolor{blue}{$a_3$} is projected onto \textcolor{red}{$u_1$} and \textcolor{green}{$u_2$},
        to obtain \textcolor{blue}{$u_3$}.
  \item<4-> \textcolor{red}{$u_1$}, \textcolor{green}{$u_2$} and \textcolor{blue}{$u_3$} are normalized.
\end{enumerate}

\end{scriptsize}

\begin{figure}
  \centering
  \includegraphics[width=0.33\textwidth]{figure_man/frame_066_delay-3s.png}
  \includegraphics[width=0.33\textwidth]{figure_man/frame_069_delay-3s.png}
  \includegraphics[width=0.33\textwidth]{figure_man/frame_089_delay-4s.png}
\end{figure}
\tiny{\url{https://commons.wikimedia.org/wiki/File:Gram-Schmidt_orthonormalization_process.gif}}

\framebreak

\begin{scriptsize}

\textbf{Given:} Three independent vectors \textcolor{red}{$a_1$}, \textcolor{green}{$a_2$}, \textcolor{blue}{$a_3$} \\
\textbf{Aim:} Vectors of an orthonormal basis \textcolor{red}{$q_1$}, \textcolor{green}{$q_2$}, \textcolor{blue}{$q_3$}

\setbeamercovered{transparent}
\begin{enumerate}
  \item<2-> \textcolor{red}{$a_1$} serves as the first vector of the orthogonal basis (\textcolor{red}{$u_1$}).
  \item<3-> \textcolor{green}{$a_2$} is projected onto \textcolor{red}{$u_1$}; projection is substracted from \textcolor{green}{$a_2$}
        to obtain \textcolor{green}{$u_2$}.
  \item \textcolor{blue}{$a_3$} is projected onto \textcolor{red}{$u_1$} and \textcolor{green}{$u_2$},
        to obtain \textcolor{blue}{$u_3$}.
  \item<4-> \textcolor{red}{$u_1$}, \textcolor{green}{$u_2$} and \textcolor{blue}{$u_3$} are normalized.
\end{enumerate}

\end{scriptsize}

\begin{figure}
  \centering
  \includegraphics[width=0.45\textwidth]{figure_man/frame_092_delay-1s.png}
  \includegraphics[width=0.45\textwidth]{figure_man/frame_095_delay-3s.png}
\end{figure}
\tiny{\url{https://commons.wikimedia.org/wiki/File:Gram-Schmidt_orthonormalization_process.gif}}

\framebreak

\begin{scriptsize}

\textbf{Given:} Three independent vectors \textcolor{red}{$a_1$}, \textcolor{green}{$a_2$}, \textcolor{blue}{$a_3$} \\
\textbf{Aim:} Vectors of an orthonormal basis \textcolor{red}{$q_1$}, \textcolor{green}{$q_2$}, \textcolor{blue}{$q_3$}

\setbeamercovered{transparent}
\begin{enumerate}
  \item<2-> \textcolor{red}{$a_1$} serves as the first vector of the orthogonal basis (\textcolor{red}{$u_1$}).
  \item<3-> \textcolor{green}{$a_2$} is projected onto \textcolor{red}{$u_1$}; projection is substracted from \textcolor{green}{$a_2$}
        to obtain \textcolor{green}{$u_2$}.
  \item \textcolor{blue}{$a_3$} is projected onto \textcolor{red}{$u_1$} and \textcolor{green}{$u_2$},
        to obtain \textcolor{blue}{$u_3$}.
  \item<4-> \textcolor{red}{$u_1$}, \textcolor{green}{$u_2$} and \textcolor{blue}{$u_3$} are normalized.
\end{enumerate}

\end{scriptsize}

\begin{figure}
  \centering
  \includegraphics[width=0.45\textwidth]{figure_man/frame_098_delay-3s.png}
  \includegraphics[width=0.45\textwidth]{figure_man/frame_118_delay-4s.png}
\end{figure}
\tiny{\url{https://commons.wikimedia.org/wiki/File:Gram-Schmidt_orthonormalization_process.gif}}

\framebreak

\begin{scriptsize}

\textbf{Given:} Three independent vectors \textcolor{red}{$a_1$}, \textcolor{green}{$a_2$}, \textcolor{blue}{$a_3$} \\
\textbf{Aim:} Vectors of an orthonormal basis \textcolor{red}{$q_1$}, \textcolor{green}{$q_2$}, \textcolor{blue}{$q_3$}

\setbeamercovered{transparent}
\begin{enumerate}
  \item<2-> \textcolor{red}{$a_1$} serves as the first vector of the orthogonal basis (\textcolor{red}{$u_1$}).
  \item<3-> \textcolor{green}{$a_2$} is projected onto \textcolor{red}{$u_1$}; projection is substracted from \textcolor{green}{$a_2$}
        to obtain \textcolor{green}{$u_2$}.
  \item<4-> \textcolor{blue}{$a_3$} is projected onto \textcolor{red}{$u_1$} and \textcolor{green}{$u_2$},
        to obtain \textcolor{blue}{$u_3$}.
  \item \textcolor{red}{$u_1$}, \textcolor{green}{$u_2$} and \textcolor{blue}{$u_3$} are normalized.
\end{enumerate}

\end{scriptsize}

\begin{figure}
  \centering
  \includegraphics[width=0.45\textwidth]{figure_man/frame_131_delay-2s.png}
  \includegraphics[width=0.45\textwidth]{figure_man/frame_134_delay-3s.png}
\end{figure}
\tiny{\url{https://commons.wikimedia.org/wiki/File:Gram-Schmidt_orthonormalization_process.gif}}

\framebreak

\begin{scriptsize}

\textbf{Given:} Three independent vectors \textcolor{red}{$a_1$}, \textcolor{green}{$a_2$}, \textcolor{blue}{$a_3$} \\
\textbf{Aim:} Vectors of an orthonormal basis \textcolor{red}{$q_1$}, \textcolor{green}{$q_2$}, \textcolor{blue}{$q_3$}

\setbeamercovered{transparent}
\begin{enumerate}
  \item<2-> \textcolor{red}{$a_1$} serves as the first vector of the orthogonal basis (\textcolor{red}{$u_1$}).
  \item<3-> \textcolor{green}{$a_2$} is projected onto \textcolor{red}{$u_1$}; projection is substracted from \textcolor{green}{$a_2$}
        to obtain \textcolor{green}{$u_2$}.
  \item<4-> \textcolor{blue}{$a_3$} is projected onto \textcolor{red}{$u_1$} and \textcolor{green}{$u_2$},
        to obtain \textcolor{blue}{$u_3$}.
  \item \textcolor{red}{$u_1$}, \textcolor{green}{$u_2$} and \textcolor{blue}{$u_3$} are normalized.
\end{enumerate}

\end{scriptsize}

\begin{figure}
  \centering
  \includegraphics[width=0.45\textwidth]{figure_man/frame_144_delay-3s.png}
  \includegraphics[width=0.45\textwidth]{figure_man/frame_149_delay-8s.png}
\end{figure}
\tiny{\url{https://commons.wikimedia.org/wiki/File:Gram-Schmidt_orthonormalization_process.gif}}

\end{vbframe}

\begin{vbframe}{QR decomposition: Example}

Calculation of $\bm{A} = \bm{Q}\bm{R}$ with $\bm{A}$ given by

\begin{footnotesize}
$$
\Amat = \begin{pmatrix*}[r]
0 & -20 & -14 \\
3 & 27 & -4 \\
4 & 11 & -2 \end{pmatrix*}
$$
\end{footnotesize}

\begin{footnotesize}
$k = 1$:
\vspace*{-0.5cm}
\begin{align*}
  \bm{u}_1 =& \bm{a}_1 = \mat{0 \\ 3 \\ 4} \\
  \bm{q}_1 =& \frac{\bm{u}_1}{\|\bm{u}_1\|} = \frac{\bm{u}_1}{\sqrt{0+9+16}}
           = \frac{1}{5} \mat{0 \\ 3 \\ 4} \\
  r_{11} =& \scp{\bm{q}_1}{\bm{a}_1} = \frac{1}{5} (0^2 + 3^2 + 4^2) = 5 \\
    r_{12} =& \scp{\bm{q}_1}{\bm{a}_2} = \frac{1}{5} (0 \cdot (-20) + 3 \cdot 27 + 4 \cdot 11) = 25 \\
        r_{13} =& \scp{\bm{q}_1}{\bm{a}_3} = \frac{1}{5} (0 \cdot (-14) + 3 \cdot (-4) + 4 \cdot (-2)) = -4
\end{align*}

\framebreak

$k = 2$:

\vspace*{-0.5cm}

\begin{align*}
  \bm{u}_2 =& \bm{a}_2 - \frac{\langle \bm{u}_1, \bm{a}_2 \rangle}{\langle \bm{u}_1, \bm{u}_1 \rangle} \bm{u}_1 \\
           =& \bm{a}_2 - \frac{125}{25} \mat{0 \\ 3 \\ 4} \\
           =& \mat{-20 \\ 12 \\ -9} \\
  \bm{q}_2 =& \frac{\bm{u}_2}{\|\bm{u}_2\|} = \frac{\bm{u}_2}{\sqrt{400+144+81}}
           = \frac{1}{25} \mat{-20 \\ 12 \\ -9} \\
               r_{22} =& \scp{\bm{q}_2}{\bm{a}_2} = \frac{1}{25} ((-20) \cdot (-20) + 12 \cdot 27 + (-9) \cdot 11) = 25 \\
           r_{23} =& \scp{\bm{q}_2}{\bm{a}_3} = \frac{1}{25} ((-20) \cdot (-14) + 12 \cdot (-4) + (-9) \cdot (-2)) = 10
\end{align*}

\framebreak

$k = 3$:

\begin{align*}
  \bm{u}_3 =& \bm{a}_3 - \frac{\langle \bm{u}_1, \bm{a}_3 \rangle}{\langle \bm{u}_1, \bm{u}_1 \rangle} \bm{u}_1
                       - \frac{\langle \bm{u}_2, \bm{a}_3 \rangle}{\langle \bm{u}_2, \bm{u}_2 \rangle} \bm{u}_2 \\
           =& \bm{a}_3 - \frac{-20}{25} \mat{0 \\ 3 \\ 4} - \frac{250}{625} \mat{-20 \\ 12 \\ -9} \\
           =& \mat{-6 \\ -6.4 \\ 4.8} \\
  \bm{q}_3 =& \frac{\bm{u}_3}{\|\bm{u}_3\|} = \frac{1}{25} \mat{-15 \\ -16 \\ 12} \\
  r_{33} =& \scp{\bm{q}_3}{\bm{a}_3} = \frac{1}{25} ((-15) \cdot (-14) + (-16) \cdot (-4) + 12 \cdot (-2)) = 10
\end{align*}
\end{footnotesize}

\framebreak


% \medskip
% $r_{13} \leftarrow \mathbf{q}_1^\top\mathbf{x}_3 = -4$ und $r_{23} \leftarrow \mathbf{q}_2^\top\mathbf{x}_3 = 10$
% $\mathbf{q}_3 \leftarrow \mathbf{x}_3 - r_{13}\mathbf{q}_1 - r_{23}\mathbf{q}_2 = \frac{2}{5}
%   \begin{pmatrix*}[r] -15 \\ -16 \\ 12\end{pmatrix*}$ \\
% $r_{33} \leftarrow \|\mathbf{q}_3\| = 10$ und $\mathbf{q}_3 \leftarrow \frac{\mathbf{q}_3}{r_{33}} =
%   \frac{1}{25} \begin{pmatrix*}[r] -15 \\ -16 \\ 12 \end{pmatrix*}$\\
% \medskip

\framebreak

This results in
$$
\mathbf{Q} = \frac{1}{25} \begin{pmatrix*}[r]
0 & -20 & -15 \\
15 & 12 & -16 \\
20 & -9 & 12 \end{pmatrix*}
\quad \text{ and } \quad \mathbf{R} = \begin{pmatrix*}[r]
5 & 25 & -4 \\
0 & 25 & 10 \\
0 & 0 & 10 \end{pmatrix*}.
$$

\end{vbframe}


\begin{vbframe}{Householder and Givens matrix}

\vspace*{-0.2cm}
\textbf{Problem in practice:}\\
$\mathbf{Q}$ often not really orthogonal when using the above algorithm due to numerical reasons.\\
\vspace*{0.2cm}
Two other methods for QR decomposition\\
\medskip
{\bf Householder matrix:}\\
\medskip
For vector $\mathbf{u}$, matrix $\mathbf{U} = \mathbf{I} - d\mathbf{uu}^\top$ is orthogonal,
if $d = 2/ \mathbf{u}^\top\mathbf{u}$.
Choose $\mathbf{u} = \mathbf{x} + s\mathbf{e}_1$ with $s = \mathbf{x}^\top\mathbf{x}$
$\quad \Rightarrow \quad \mathbf{Ux} = - s\mathbf{e}_1$.\\
\medskip
Successive elimination of column elements yields QR decomposition.\\
\medskip
{\bf Givens matrix:}\\
\medskip
Similar to Householder, but orthogonal transformations that eliminate an element of a column vector each, and change a second vector. \\
\medskip

\begin{footnotesize}
For details see Carl D.\ Meyer \emph{Matrix Analysis and Applied Linear Algebra}.
\end{footnotesize}

\end{vbframe}

\begin{vbframe} {Properties of QR decomposition}

\begin{itemize}
\item Splitting a matrix into an orthogonal matrix $\mathbf{Q}$ and $\mathbf{R}$
% \item $\Amat$ is split into a lower triangle matrix $\mathbf{L}$ and a lower triangle matrix $\mathbf{U}$.
\item Gram-Schmidt process is numerically unstable, but can be extended and numerically stabilized
\item \textbf{Existence:} Decomposition exists for each $n \times n$ matrix and can be extended to general $m \times n, m \ne n$ matrices
\item Runtime behavior: Numerical stable solution of Householder transformation or Givens rotation comes along with higher effort:
\begin{itemize}
\item Decomposition of $n \times n$ matrix using Householder transformation: $\approx \frac{2}{3}n^3$ multiplications
\item Forward and back substitution: $n^2$
\end{itemize}
% $\to$ Gesamtaufwand von $\order(n^3)$
% \item Rechnung für Zerlegung kann auf Speicher der Matrix $\Amat$ durchgeführt werden \\
% $\to$ kein zusätzlicher Speicher nötig
\end{itemize}

\end{vbframe}

\begin{vbframe}{Comparison of methods}

\begin{table}
\centering
\begin{tabular}{c|c|c|c}
Procedure & $\mathbf{A}$ & \# Multiplications & Stability \\
\hline
LU & regular & $\approx \frac{1}{3} n^3$ & yes, by pivoting \\
Cholesky & p.d. & $\approx \frac{1}{6}n^3$ & yes \\
QR (Gram Schmidt) & - & $\approx 2n^3$ & no \\
QR (Householder) & - & $\approx \frac{2}{3}n^3$ & yes \\
\end{tabular}
\end{table}


\end{vbframe}

\begin{vbframe}{QR decomposition for $m \times n$ matrices}

General $m \times n, m \ge n$ matrices can be decomposed as well when using QR decomposition.

$$
\Amat = \bm{QR} = \bm{Q} \begin{bmatrix} \bm{R}_1 \\ \mathbf{0} \end{bmatrix} = \begin{bmatrix} \bm{Q}_1 & \bm{Q}_2 \end{bmatrix} \begin{bmatrix} \bm{R}_1 \\ \mathbf{0} \end{bmatrix} = \bm{Q}_1 \bm{R}_1
$$

$\bm{Q}_1 \in \R^{m \times n}, \bm{Q}_2 \in \R^{m\times(m-n)}$ with orthogonal columns, and $\bm{R} \in \R^{n \times n}$ upper triangular matrix.

\lz

$\bm{Q}_1 \times \bm{R}_1$ is known as a \textbf{reduced} QR decomposition.

\end{vbframe}


\endlecture
\end{document}







