\documentclass[11pt,compress,t,notes=noshow, xcolor=table]{beamer}
\input{../../style/preamble}

% latex-math includes as needed
\input{../../latex-math/basic-math}
\input{../../latex-math/basic-ml}

% Lecture title always has to be there
\title{Algorithms and Data Structures}

\begin{document}

\titlemeta{Big O}{Properties \& Examples of Big O}
{figure_man/oruntime}
{
  \item Properties of Big O
  \item Know how to determine the runtime
  \item Complexity classes
}

%\lecture{CIM1 Statistical Computing}

\begin{vbframe}{Properties}

Be $f, g, h, f_i, g_i: X \to \R$, $c \ge 0$.

\begin{enumerate}
  \item Constants: $f \in \order(c g)$ is equivalent to $f \in \order(g)$. In particular: $f \in \order(c)$ is equivalent to $f \in \order(1)$ (Constant runtime)
  \item Transitivity: If $f \in \order(g)$ and $g \in \order(h)$ then $f \in \order(h)$
  \item Products: $f_{1} \in \order(g_{1}){\text{ and }}f_{2} \in \order(g_{2})\,\Rightarrow f_{1}f_{2} \in \order(g_{1}g_{2})$
  \item Sums: $f_{1}\in \order(g_{1}){\text{ and }}f_{2} \in \order(g_{2})\,\Rightarrow f_{1}+f_{2} \in \order(|g_{1}|+|g_{2}|)$
  % \item factorization: If $f \in \order(g \cdot h)$, then $f \in g \cdot \order(h)$
  %\item $f \cdot \order(g) \subset \order(f \cdot g)$
\end{enumerate}

\vfill

\framebreak

Particularly important for determining the runtime of an algorithm:
\begin{itemize}
   \item If a function is the sum of several functions, the fastest growing function determines the order of the sum of functions.
  \item If $f$ is a product of several factors, constants can be neglected.
\end{itemize}

\lz

\textbf{Example 1: }

The complexity of the function $f(n) = n \log n + 3 \cdot n^3$ can be determined quickly: the fastest growing function is $3 \cdot n^3$, multiplicative constants can be neglected. So

$$
f(n) \in \order(n^3)
$$

\end{vbframe}

\begin{vbframe}{Other examples}

\textbf{Example 2:}

$$f(n) = 10 \log(n) + 5 (\log(n))^3 + 7n + 3n^2 + 6n^3$$

\begin{itemize}
\item The fastest growing summand is $6n^3$
\item Constants can be neglected
\item $\Rightarrow f(n) \in \order(n^3)$
\end{itemize}

\lz

\textbf{Example 3:}

$$ g(n) = n^2 \cdot \exp(n) $$
\begin{itemize}
\item $\Rightarrow g(n) \in \order(n^2 \cdot \exp(n))$
\end{itemize}

\end{vbframe}

\begin{vbframe}{Determining the runtime}
How fast a function runs depends on the different statements that are executed.
$$
total\_time = time(statement_1) + time(statement_2) + \ldots + time(statement_k)
$$

If each statement is a simple base operation, the time for each
statement is constant and the total runtime is also constant: $\order(1)$.

\framebreak

\begin{block}{If-else}
\begin{verbatim}
if (cond) {
  block1 # sequence of statements
} else {
  block2 # sequence of statements
}
\end{verbatim}

\begin{itemize}
  \item Either \code{block1} \textbf{or} \code{block2} is executed
  \item The worst case is the slower one of the two options:
  $$
  \max(time(block1), time(block2))
  $$
\end{itemize}
\end{block}

\framebreak

\begin{block}{Loops}
\begin{verbatim}
for (i in 1:n) {
  block # sequence of statements
}
\end{verbatim}

\begin{itemize}
  \item We consider $n$ as part of our input size (e.g., number of elements in a list).
  \item The loop is executed $n$ times.
  \item If we assume that the statements are $\order(1)$, then the total runtime is: $n \cdot \order(1) = \order(n)$.
\end{itemize}
\end{block}

\framebreak

\begin{block}{Nested loops}
\begin{verbatim}
for (i in 1:n) {
  for (j in 1:m) {
    block # sequence of statements
  }
}
\end{verbatim}

\begin{itemize}
  \item Let $m$, $n$ be part of our input size (e.g. number of rows/columns of a matrix).
  \item The outer loop is executed $n$ times.
  \item At each iteration of $i$ the inner loop is executed $m$ times.
  \item Thus the statements are executed $n \cdot m$ times in total and the complexity is $\order(n \cdot m)$.
\end{itemize}
\end{block}

\framebreak

\begin{block}{Statements with function calls}
\begin{itemize}
  \item When a statement calls a function, the complexity of the function must be included in the calculation.
  \item This also holds for loops:\\
  \begin{verbatim}
  for (i in 1:n) {
    g(i)
  }
  \end{verbatim}
  If $g \in \order(n)$, the runtime of the loop is $\order(n^2)$.
\end{itemize}

\end{block}

\end{vbframe}
\begin{vbframe}{Examples (continued)}

\textbf{Example 4: } Bubble sort algorithm

\vspace*{0.2cm}

The bubble sort is an algorithm that sorts the elements of a (numeric) vector of length $n$ in ascending order.

\vspace*{0.4cm}

\begin{minipage}{.45\textwidth}
\begin{footnotesize}
\begin{verbatim}
for (k in n:2) {
  for (i in 1:(k - 1)) {
    if (x[i] > x[i + 1]) {
      # swap elements
      s = x[i]
      x[i] = x[i + 1]
      x[i + 1] = s
    }
  }
}
\end{verbatim}
\end{footnotesize}
\end{minipage}\hfill
\begin{minipage}{.45\textwidth}
\includegraphics[width= 0.6\textwidth]{figure_man/bubble-sort-1.png}\\
\begin{footnotesize}
\url{http://teerexie.blogspot.com/}
\end{footnotesize}

\end{minipage}

\framebreak

\begin{itemize}
\item The inner loop depends on the outer loop and is executed $i = n - 1$, then $i = n - 2$, ... and finally $i = 1$ times.
\item According to the sum of natural numbers (Carl Friedrich Gauss) the inner loop is executed $\sum_{i = 1}^{n - 1} i = \frac{(n - 1)n}{2} = \frac{n^2 - n}{2}$ times.
\item The operations in the \texttt{if} statement are operations with constant runtime.
\end{itemize}

The total runtime is therefore
$$\frac{n^2 - n}{2} \cdot\order(1) = \order\left(\frac{n^2 - n}{2}\right) = \order(n^2)$$

\framebreak

\textbf{Example 5: } The multiplication of two matrices $\mathbf{A} \in \R^{m \times n}, \mathbf{B} \in \R^{n \times p}$ has a runtime of $\order(mpn)$:

\begin{itemize}
\item $m \cdot p$ scalar products
\item For each scalar product: $n$ multiplications and $n - 1$ additions
\item $\rightarrow$ $m \cdot p \cdot (n + (n - 1))$ operations
\end{itemize}

\begin{center}
\includegraphics[width= 0.3\textwidth]{figure_man/matrix_multiplication.png}\\
\begin{footnotesize}
\url{https://commons.wikimedia.org/wiki/File:Matrix\_multiplication\_diagram\_2.svg}
\end{footnotesize}
\end{center}


\framebreak
The Coppersmith-Winograd algorithm allows matrix multiplication of two $n\times n$ matrices in $\order(n^{2.373})$. A lower bound for the complexity of the matrix multiplication is $n^2$, since each of the $n^2$ elements of the output matrix must be generated.

\vfill
\begin{footnotesize}
More about \href{https://en.wikipedia.org/wiki/Computational_complexity_of_mathematical_operations}{\color{blue}\underline{Computational complexity of mathematical operations}}
\end{footnotesize}

\framebreak

\begin{verbatim}
multiplyMatrices = function(n) {
  A = matrix(runif(n^2), n, n)
  B = matrix(runif(n^2), n, n)

  return(A %*% B)
}
\end{verbatim}

\vspace{-0.2cm}
\begin{center}
\includegraphics[height = 0.5\textheight, width= 0.6\textwidth]{figure_man/runtime_matmult.pdf}
\end{center}

\framebreak

If possible: Avoid matrix multiplication!
\begin{verbatim}
n = 1000
A = matrix(runif(n), n, n)
B = matrix(runif(n), n, n)
y = c(runif(n))

system.time(A %*% B %*% y)
## user system elapsed 
## 0.72 0.00 0.73

system.time(A %*% (B %*% y))
## user system elapsed 
## 0.00 0.00 0.03
\end{verbatim}

\begin{verbatim}
n = 1000
A = matrix(rnorm(n), n, n) + diag(1, nrow = n)
b = rnorm(n)

# solving Ax = b
system.time(solve(A) %*% b)  # A^{-1} %*% b
## user system elapsed 
## 0.96 0.01 0.05

system.time(solve(A, b)) # direct solution of the LES
## user system elapsed 
## 0.0 0.2 0.0
\end{verbatim}


% \lz
%
% \textbf{Beispiel 5: }
%
% Lösen eines LGS $Ax = b$ mit $\mathbf{A} \in \R^{m \times n}, \mathbf{b} \in \R^{m}$ mit Gauss

\framebreak

\textbf{Example 6:}

In mathematics one is interested in the estimation of error terms for approximations.

Using Taylor's theorem a $m$-times differentiable function $f$ at point $x = x_0$ can be defined as follows:

\begin{footnotesize}
\begin{eqnarray*}
f(x) &=& f(x_0) + \frac{f'(x_0)}{1!}(x - x_0) +  \frac{f'(x_0)}{2!}(x - x_0)^2 + ... +  \frac{f^{(m)}(x_0)}{m!}(x - x_0)^m \\
&+& \order(|x - x_0|^{m + 1}), \quad x\to x_0.
\end{eqnarray*}
\end{footnotesize}

\begin{itemize}
\item The more $x$ approaches $x_0$, the better the Taylor polynomial approximates $f$ at point $x$.
\item The higher the order $m$ of the Taylor polynomial, the better the approximation for $x \to x_0$.
\end{itemize}


\framebreak

For example, consider the exponential function as \textbf{Taylor series}

$$
\exp(x) = \sum_{i=0}^\infty \frac{x^i}{i!}
$$

$\exp(x)$ approximated at the point $x=0$

$$
\exp(x) = 1 + x + \frac{x^2}{2!} + \order(x^3) \text{ for } x \rightarrow 0
$$

In this way, it becomes clear that the error does not become greater than
$M \cdot x^3$ when $x$ approaches $0$.

\framebreak

\textbf{Example 7:}

The complexity of the \textbf{binary search} is visualized by a tree representation.

\begin{center}
  \includegraphics[width=0.3\textwidth]{figure_man/binarysearch.png}
\end{center}

For an array of length $n$, the search tree has a height of $\log_2(n)$. After a maximum of $\log_2(n)$ comparisons, the searched element is found. The complexity of the binary search is $\order(\log n)$.

\framebreak

\textbf{Example 8:}

\footnotesize
The \textbf{Fibonacci sequence} is a series of numbers where each number is the sum of the two preceding ones, starting with 1. The sequence thus begins as:

$1, 1, 2, 3, 5, 8, 13, 21, 34, ...$

\begin{scriptsize}
\begin{verbatim}
fib = function(n) {
  if (n <= 2L)
    return(1L)
  return(fib(n - 2) + fib(n - 1))
}

fib_table = microbenchmark(fib(5), fib(10), fib(20), fib(21), times = 500L)
print(xtable(summary(fib_table), digits = 0), booktabs=TRUE, 
    caption.placement="top", size="\\fontsize{8pt}{9pt}\\selectfont")
\end{verbatim}
\begin{table}[ht]
  \centering
  \begingroup\fontsize{8pt}{9pt}\selectfont
  \begin{tabular}{rlrrrrrrr}
    \toprule
   & expr & min & lq & mean & median & uq & max & neval \\ 
    \midrule
  1 & fib(5) & 2 & 2 & 99 & 3 & 4 & 47817 & 500 \\ 
    2 & fib(10) & 27 & 29 & 32 & 30 & 33 & 88 & 500 \\ 
    3 & fib(20) & 3611 & 3733 & 4164 & 3842 & 4052 & 10118 & 500 \\ 
    4 & fib(21) & 5861 & 6047 & 6926 & 6227 & 6476 & 49636 & 500 \\ 
     \bottomrule
  \end{tabular}
  \endgroup
  \end{table}
\end{scriptsize}

\framebreak

$Fibonacci(n) \in \order(2^n)$ (exponential runtime) \\

\lz

\textbf{Informal proof:}

$$
\texttt{Fibonacci(n)} = \underbrace{\texttt{Fibonacci(n - 1)}}_{T(n-1)} \underbrace{+}_{\order(1)} \underbrace{\texttt{Fibonacci(n - 2)}}_{T(n-2)}
$$

This results in a runtime of $T(n) = T(n-1) + T(n-2) + \order(1)$ for $n >1$.

\lz

The function is executed twice in each step.

\begin{eqnarray*}
  T(n) &=& T(n-1) + T(n-2)  \\
  &=& T(n-2) + T(n-3) + T(n-3) + T(n-4) = \ldots
\end{eqnarray*}

\framebreak

\begin{center}
\includegraphics[width=0.5\textwidth]{figure_man/fibonacci.png}
\end{center}

By simply "counting" the nodes of this recursion tree you can determine the exact number of operations. \\
$\to$ Worst case runtime $\order(2^n)$.

\framebreak


\textbf{Variations of Fibonacci(n): Iterative}
\begin{scriptsize}
\begin{verbatim}
fib2 = function(n) {
  a = 0; b = 1
  if (n <= 2)
    return(1)
  for (i in seq_len(n-1L)) {
    tmp = b;  b = a + b; a = tmp
  }
  return(b)
}
\end{verbatim}
\end{scriptsize}
This is $\order(n)$ (if we, incorrectly, assume addition is constant in n).

\framebreak
\begin{scriptsize}
\begin{verbatim}
fib2_table = microbenchmark(fib2(10), fib2(20), fib2(40), fib2(80), 
                            fib2(160), times = 5000L)

print(xtable(summary(fib2_table), digits = 0), booktabs=TRUE, 
    caption.placement="top", size="\\fontsize{8pt}{9pt}\\selectfont")
\end{verbatim}

\begin{table}[ht]
  \centering
  \begingroup\fontsize{8pt}{9pt}\selectfont
  \begin{tabular}{rlrrrrrrr}
    \toprule
   & expr & min & lq & mean & median & uq & max & neval \\ 
    \midrule
  1 & fib2(10) & 1 & 1 & 2 & 1 & 2 & 37 & 5000 \\ 
    2 & fib2(20) & 1 & 2 & 2 & 2 & 2 & 24 & 5000 \\ 
    3 & fib2(40) & 2 & 2 & 4 & 2 & 3 & 6755 & 5000 \\ 
    4 & fib2(80) & 3 & 3 & 4 & 3 & 4 & 26 & 5000 \\ 
    5 & fib2(160) & 5 & 5 & 7 & 6 & 7 & 49 & 5000 \\ 
     \bottomrule
  \end{tabular}
  \endgroup
  \end{table}
\end{scriptsize}
Time measurement becomes imprecise since ``for loops'' are not that slow in \texttt{R} due to JIT compilation. Hence we are using doubles here as a lazy trick to generate large fibonacci numbers. An alternative to generate large integers would be to use the int64 package.\\





\framebreak

\textbf{Variations of Fibonacci(n): In \texttt{C}}

\begin{scriptsize}
\begin{verbatim}
library(inline)
fib3 = cfunction(signature(n="integer"), language="C",
         convention=".Call", body = '
         int nn = INTEGER(n)[0];
         SEXP res;
         PROTECT(res = allocVector(INTSXP, 1));
         INTEGER(res)[0] = 1;
         int a = 0; int b = 1;
         for (int i=0; i<nn-1; i++) {
         int tmp = b;
         b = a + b;
         a = tmp;
         }
         INTEGER(res)[0] = b;
         UNPROTECT(1);
         return res;
         ')
\end{verbatim}
\end{scriptsize}
See how ugly the \texttt{C} interface is?
\framebreak

%\textbf{Variations of fib(n): In C}
\begin{scriptsize}
\begin{verbatim}
fib3_table = microbenchmark(fib3(20L), fib3(40L),times = 5000L)

print(xtable(summary(fib3_table), digits = 0), booktabs=TRUE, 
    caption.placement="top", size="\\fontsize{8pt}{9pt}\\selectfont")
\end{verbatim}

\begin{table}[ht]
  \centering
  \begingroup\fontsize{8pt}{9pt}\selectfont
  \begin{tabular}{rlrrrrrrr}
    \toprule
   & expr & min & lq & mean & median & uq & max & neval \\ 
    \midrule
  1 & fib3(20L) & 300 & 400 & 465 & 400 & 500 & 13900 & 5000 \\ 
    2 & fib3(40L) & 300 & 400 & 479 & 400 & 500 & 21200 & 5000 \\ 
     \bottomrule
  \end{tabular}
  \endgroup
  \end{table}
\end{scriptsize}

This is both $\order(n)$ ... See the difference? Actually, you do not see anything as the function
is so fast, we would need to calculate with bigints to really see the $\order(n)$!

\framebreak

\textbf{Variations of Fibonacci(n): \texttt{C++}-version}
\begin{verbatim}
library(Rcpp)
fib4 = cppFunction('int fibonacci(const int x) {
         if (x <= 2) return(1);
         return (fibonacci(x - 1)) + fibonacci(x - 2);
}
' )
\end{verbatim}

Much nicer \texttt{C++}-Interface with Rcpp.

\framebreak

%\textbf{Variations of fib(n): C++-version}
%\begin{verbatim}
%microbenchmark(fib3(20L), fib4(20L), fib3(40L), fib4(40L))
%\end{verbatim}

%\framebreak


\textbf{Variations of Fibonacci(n): Matrix power-exponentiation}

\begin{footnotesize}
\begin{verbatim}
library(expm)
fib5 = function(n) {
  A = matrix(c(1, 1, 1, 0), 2, 2)
  B = A%^%n
  B[1, 2]
}
\end{verbatim}
\end{footnotesize}

How does \pkg{fib5()} work?

\begin{align*}
  \bm{A} &= \mat{1 & \color{red}{1} \\ 1 & 0} \quad
  \bm{A}^2 = \mat{2 & \color{red}{1} \\ 1 & 1} \quad
  \bm{A}^3 = \mat{3 & \color{red}{2} \\ 2 & 1} \quad
  \bm{A}^4 = \mat{5 & \color{red}{3} \\ 3 & 2} \\
  \bm{A}^5 &= \mat{8 & \color{red}{5} \\ 5 & 3} \quad
  \bm{A}^6 = \mat{13 & \color{red}{8} \\ 8 & 5} \quad
  \bm{A}^7 = \mat{21 & \color{red}{13} \\ 13 & 8} \quad
  \hdots
\end{align*}

\framebreak


\textbf{Matrix power-exponentiation}

What does \verb|A %^% n| do? \\
Computes the n-th power of a matrix corresponding to $n-1$ matrix multiplications
(\verb|A^n| only computes element wise powers).

The algorithm uses $\order(log_2(k))$ matrix multiplications.

\lz
\textbf{Exponentiation by squaring:}

$$
  x^n =
  \begin{cases}
    x(x^2)^{\frac{n-1}{2}} & \text{if n is odd} \\
    (x^2)^{\frac{n}{2}}    & \text{if n is even}
  \end{cases}
$$

\framebreak

\textbf{Exponentiation by squaring}

Implemented as a recursive algorithm:

\begin{scriptsize}
\begin{verbatim}
exp.by.squaring = function(x, n) {
  if(n<0) {
    return(exp.by.squaring(1 / x, -n))
  } else if(n==0){
    return(1)
  } else if(n==1){
    return(x)
  } else if(n%%2 == 0){
    return(exp.by.squaring(x^2, n/2))
  } else {
    return(x * exp.by.squaring(x^2, (n-1)/2))
  }
}
exp.by.squaring(2,5)
## [1] 32
\end{verbatim}
\end{scriptsize}

% Time can not be measured via microbenchmark since it is too fast.
\framebreak

%\textbf{Variations of Fibonacci(n): Matrix power-exponentiation}
%\begin{verbatim}
%microbenchmark(fib5(7), fib5(25L), fib5(49), times = 1000)
%\end{verbatim}


\framebreak


\textbf{Example 9:} The \textbf{Traveling Salesman Problem} (TSP) is the problem of planning a route through all locations in such a way that

\begin{itemize}
\item The entire route is as short as possible,
\item The first location is equal to the last location.
\end{itemize}

\vspace*{-0.2cm}

\begin{center}
  \includegraphics[width=0.3\textwidth]{figure_man/tsp.png}~~\includegraphics[width=0.4\textwidth]{figure_man/weighted-graph.png}
\end{center}
\vspace*{-0.2cm}

\begin{footnotesize}
Left: Route through places in Germany (\url{https://de.wikipedia.org/wiki/Problem_des_Handlungsreisenden}) \\
Right: Weighted graph (\url{https://www.chegg.com/})
\end{footnotesize}

\framebreak

Exact algorithms with long runtime exist

\begin{itemize}
  \item Brute force search (Calculate lengths of all possible round trips and choose shortest): $\order(n!)$
  \item Dynamic Programming (Held-Karp algorithm): $\order(n^22^n)$
\end{itemize}

and heuristic algorithms with shorter runtime, which do not guarantee an optimal solution, e.g.

\begin{itemize}
  \item Nearest-Neighbor heuristics: $\order(n^2)$
  %\item Minimum-Spanning-Tree-Heuristic: $\order(n^2\log(n))$
\end{itemize}

The TSP problem is \textbf{NP-complete}.

\end{vbframe}


\begin{vbframe}{Complexity classes}

In theoretical computer science, problems are divided into complexity classes. For an input size $n$ a distinction is made between

\begin{itemize}
\item \textbf{P}: Problems solvable in polynomial runtime ($\order(n^k), k\ge 1$)
\item \textbf{NP} (\textbf{N}on-deterministic \textbf{P}olynomial time): Problems from \textbf{P} and problems that cannot be solved in polynomial time;\\ NP problems can only be solved with a non-deterministic turing machine in an acceptable time (hence the name)
\item \textbf{NP-complete}: All problems from NP can be traced back to this problem
\end{itemize}

It has not yet been proven that P $\ne$ NP holds.

\end{vbframe}

\endlecture
\end{document}